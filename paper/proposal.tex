\documentclass{beamer}

\usetheme{Warsaw}
\usecolortheme{default}
\setbeamercolor{frametitle}{bg=blue, fg=white}

\title{Minimum Wage and Occupational Dynamics in Russia}
\subtitle{Evidence from RLMS 1994-2024}
\author{
Albert Aina, \\ 
Yahaya Ali \\
\vspace{0.2cm}
\textit{Academic Adviser: Associate Prof. Dmitry Rudenko}
}
\date{17 December, 2025}

\begin{document}

%------------------------------------------------
\begin{frame}
\titlepage
\begin{figure}
    \centering
    \includegraphics[width=0.3\linewidth]{Hse Logo.png}
\end{figure}
\end{frame}

%------------------------------------------------
\begin{frame}{Introduction}
\framesubtitle{What Is the Minimum Wage?}
The minimum wage is a government-mandated wage floor designed to protect workers from excessively low pay.

\begin{figure}
  \centering
  \includegraphics[width=0.6\linewidth]{minimum-wage-in-russia-2011-2025.png} 
\end{figure}
\end{frame}

%------------------------------------------------
\begin{frame}{Introduction}
\framesubtitle{Occupational Dynamics and Wage Floors}
In Russia, minimum wage serves as a key labor policy instrument aimed at:
\begin{itemize}
  \item Ensuring basic income security
  \item Preventing in-work poverty
  \item Shaping employment conditions across sectors and regions
\end{itemize}

\textbf{Beyond social protection}, minimum wage policy can influence:
\begin{itemize}
  \item Employers' hiring and task allocation decisions
  \item Workers' occupational choices and sectoral mobility
  \item The distribution of employment across occupations
\end{itemize}
\end{frame}

%------------------------------------------------
\begin{frame}{Average Monthly Wages by Salary Group} 
\framesubtitle{Distribution of Earnings in Russia}
\centering
\includegraphics[width=0.7\linewidth]{Average monthly wage in Russia in April 2023 by 10-percent salary group.png}
\end{frame}

%------------------------------------------------
\begin{frame}{Background}
\framesubtitle{Evolution of Economic Views}
Minimum wage policy has long been debated in economic theory.

\begin{itemize}
  \item Classical models predicted negative employment effects
  \item Since the 1990s, empirical research shows heterogeneous outcomes
  \item Recent studies emphasize institutions and market structure
\end{itemize}

\centering
\includegraphics[width=0.5\linewidth]{Standard-min-wage-graph1.png} \\
\footnotesize Classical Economics View of Minimum Wage
\end{frame}

%------------------------------------------------
\begin{frame}{Background}
\framesubtitle{Russian Institutional Context}
\begin{itemize}
  \item Minimum wages are implemented at the regional level
\end{itemize}

\centering
\includegraphics[width=0.5\linewidth]{monthly-minimum-wage-in-russia-and-its-major-cities-2025.png} \\  

\begin{itemize}
  \item Regions differ in economic structure and composition
  \item Formal and informal employment coexist
  \item Major reforms: 2005, 2007, 2009, 2016-2018
\end{itemize}
\end{frame}

%------------------------------------------------
\begin{frame}{Literature Review}
\framesubtitle{Existing Evidence}

\textbf{International Evidence:}
\begin{itemize}
\item Card \& Krueger (1994): Small employment effects in US
\item Neumark \& Wascher (2007): Negative effects in some contexts
\item Autor et al. (2016): Task-based vulnerability to labor shocks
\end{itemize}

\textbf{Russian Studies:}
\begin{itemize}
\item Lukiyanova (2010): MW compressed wage inequality
\item Gimpelson \& Kapeliushnikov (2011): Weak labor institutions
\item Regional MW studies limited to wage effects
\end{itemize}

\textbf{Gap:} No occupation-level analysis in Russian context
\end{frame}

%------------------------------------------------
\begin{frame}{Research Gap and Contribution}
\framesubtitle{Why This Study Matters}

\textbf{Research Gaps:}
\begin{itemize}
  \item Limited evidence on occupational-level effects
  \item Understudied regional heterogeneity
  \item Few studies connect MW with occupational mobility
\end{itemize}

\vspace{0.3cm}
\textbf{Original Contribution:}
\begin{itemize}
  \item \textbf{Pioneers} occupation-level analysis of MW effects in Russia's transition economy
  \item \textbf{Leverages} regional natural experiment for causal identification
  \item \textbf{Integrates} task-based approaches with institutional analysis
  \item \textbf{Delivers} policy-relevant evidence for targeted regional interventions
\end{itemize}
\end{frame}

%------------------------------------------------
\begin{frame}{Research Questions}
\framesubtitle{Primary Focus}

\textbf{Main Question:}
How do minimum wage changes affect occupational structure across Russian regions?

\vspace{0.2cm}
\textbf{Theoretical Motivation:} MW may cause occupational substitution, upgrading, or regional sorting

\vspace{0.3cm}
\textbf{Specific Questions:}
\begin{enumerate}
  \item Which occupations cluster at minimum wage thresholds?
  \item How do adjustment mechanisms differ by sector?
  \item What are the regional patterns of occupational change?
\end{enumerate}
\end{frame}

%------------------------------------------------
\begin{frame}{Research Hypotheses}
\framesubtitle{Testable Predictions}

\begin{columns}[T]
\begin{column}{0.48\textwidth}
\textbf{H1: Occupational Incidence}
\begin{itemize}
  \item Low-skill service occupations show highest MW concentration
  \item Concentration patterns differ by region
\end{itemize}

\vspace{0.3cm}
\textbf{H2: Adjustment Mechanisms}
\begin{itemize}
  \item Formal firms: Task reorganization
  \item Informal sector: Non-compliance
\end{itemize}
\end{column}

\begin{column}{0.48\textwidth}
\textbf{H3: Regional Heterogeneity}
\begin{itemize}
  \item High-MW regions: Occupational upgrading
  \item Border regions: Cross-regional mobility
\end{itemize}

\vspace{0.3cm}
\textbf{Theoretical Basis:}
\begin{itemize}
  \item Neoclassical labor demand model
  \item Task-based occupational approach
  \item Russian institutional context
\end{itemize}
\end{column}
\end{columns}
\end{frame}

%------------------------------------------------
\begin{frame}{Data Source}
\framesubtitle{Russian Longitudinal Monitoring Survey (RLMS)}

\begin{itemize}
\item Household panel: 1994-2024 (30 years)
\item 171,965 household-year observations
\item Nationally representative with regional identifiers
\item Occupational codes (3-digit OKZ)
\end{itemize}

\vspace{0.3cm}
\textbf{Sample Selection:}
\begin{itemize}
\item Working-age population (18-65)
\item Wage and salary workers
\item Complete occupation information
\end{itemize}

\textbf{Limitations:}
\begin{itemize}
\item Underrepresents informal sector
\item Occupational codes changed over time
\item Wage reporting may have measurement error
\end{itemize}
\end{frame}

%------------------------------------------------
\begin{frame}{Key Variables}
\framesubtitle{RLMS Data Structure}

\begin{table}[ht]
\centering
\scriptsize
\begin{tabular}{ll}
\toprule
\textbf{Variable} & \textbf{RLMS Code} \\
\midrule
Monthly wage & C4 \\
Occupation & C7\_1 \\
Employment status & C1 \\
Sector & C2 \\
Firm type & C3 \\
Formal status & C8 \\
Hours worked & C5 \\
Region & REGION \\
\bottomrule
\end{tabular}
\end{table}

\begin{itemize}
\item Additional controls: Age, gender, education
\item Regional economic indicators
\item Time dummies for economic cycles
\end{itemize}
\end{frame}

%------------------------------------------------
\begin{frame}{Empirical Strategy}
\framesubtitle{Three-Stage Approach}

\textbf{Stage 1: Descriptive Mapping}
\[
\text{MW Concentration}_{ort} = \frac{Emp_{ort,w \leq MW}}{Emp_{rt,w \leq MW}}
\]

\vspace{0.3cm}
\textbf{Stage 2: Difference-in-Differences}
\[
Y_{irt} = \beta_0 + \beta_1 (\text{Post}_t \times \text{HighMW}_r) + \text{Controls}
\]

\vspace{0.3cm}
\textbf{Stage 3: Border Analysis}
\[
Y_i = \alpha + \beta \cdot \mathbf{1}\{\text{MW}_r > \text{MW}_{r'}\} + \epsilon_i
\]
\end{frame}

%------------------------------------------------
\begin{frame}{Methodology}
\framesubtitle{Identification and Robustness}

\textbf{Identification Strategy:}
\begin{itemize}
\item Exploit regional MW differences post-2007
\item Temporal changes in federal MW
\item Interaction of region and time effects
\end{itemize}

\vspace{0.3cm}
\textbf{Statistical Approach:}
\begin{itemize}
\item Regional and year fixed effects
\item Clustered standard errors (region-occupation)
\item Dynamic event study design
\end{itemize}

\textbf{Robustness Checks:}
\begin{itemize}
\item Alternative wage measures (C4 vs G1\_1)
\item Different MW bite thresholds (25\%, 30\%, 35\%)
\item Placebo tests with false reform dates
\end{itemize}
\end{frame}

%------------------------------------------------
\begin{frame}{Expected Findings}
\framesubtitle{Preliminary Expectations}

\textbf{Based on literature and descriptive patterns:}
\begin{itemize}
\item 5-8\% of workers directly affected by MW (higher in low-wage regions)
\item Routine occupations (retail, cleaning) most concentrated at MW
\item Public sector shows larger spillover effects due to wage grids
\item Border regions show occupational mobility toward lower-MW areas
\item Informal employment increases in high-MW regions for affected occupations
\end{itemize}

\vspace{0.3cm}
\textbf{Policy implication:} Targeted MW adjustments needed by occupation/region
\end{frame}

%------------------------------------------------
\begin{frame}{Expected Contributions}
\framesubtitle{Academic Value}

\begin{itemize}
\item First comprehensive occupation-level analysis for Russia
\item Novel evidence on regional heterogeneity in transition economies
\item Methodological: Integrating occupational mobility with MW analysis
\item Advances understanding of institutional mediation in labor markets
\end{itemize}

\vspace{0.3cm}
\textbf{Policy Relevance:}
\begin{itemize}
\item Evidence for targeted social protection to vulnerable occupations
\item Guidelines for optimal regional differentiation of MW policies
\item Sector-specific enforcement strategies for labor inspectorates
\end{itemize}
\end{frame}

%------------------------------------------------
\begin{frame}{Research Timeline}
\framesubtitle{5-Month Schedule}

\textbf{Month 1: Preparation}
\begin{itemize}
\item Literature review completion
\item RLMS data cleaning \& preparation
\end{itemize}

\textbf{Month 2: Descriptive Analysis}
\begin{itemize}
\item Occupational incidence mapping
\item Regional heterogeneity patterns
\end{itemize}

\textbf{Month 3: Causal Analysis}
\begin{itemize}
\item Difference-in-Differences estimation
\item Border discontinuity design
\end{itemize}

\textbf{Month 4: Writing \& Extensions}
\begin{itemize}
\item Thesis drafting
\item Heterogeneity analysis
\end{itemize}

\textbf{Month 5: Completion}
\begin{itemize}
\item Final revisions
\item Presentation preparation
\item Defense
\end{itemize}
\end{frame}

%------------------------------------------------
\begin{frame}{Conclusion}
This study provides first analysis of MW effects on occupational dynamics in Russia.

\textbf{Key Innovations:}
\begin{itemize}
\item Occupational-level analysis rather than aggregate effects
\item Regional natural experiment (post-2007 differentiation)
\item Long-term dynamics (1994-2024)
\item Multiple adjustment mechanisms
\item Policy-relevant regional focus
\end{itemize}

\vspace{0.3cm}
\textbf{Significance:} Provides evidence for optimizing Russia's regional MW policy while advancing understanding of occupational adjustment mechanisms in transition economies.
\end{frame}

%------------------------------------------------
\begin{frame}
\centering
{\Huge Thank You}

\vspace{0.5cm}

\texttt{aaina@edu.hse.ru} \\
\texttt{asyahaya@edu.hse.ru} \\ [0.3cm]
\colorbox{yellow}{\texttt{drudenko@hse.ru}}

\end{frame}

\end{document}